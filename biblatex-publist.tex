%%%%%%%%%%%%%%%%%%%%%%%%%%%%%%%%%%%%%%%%%%%%%%%%%%%%%%%%%%%%%%%%%%%%%%%%%%%%%%%%%%%%%%%%%%%%%%%%%%%
%% File biblatex-publist.tex
%%
%% Manual of the biblatex-publist package.
%%
%% This file is part of the biblatex-publist package.
%%
%% Author: Juergen Spitzmueller <juergen.spitzmueller@univie.ac.at>
%%
%% This work may be distributed and/or modified under the
%% conditions of the LaTeX Project Public License, either version 1.3
%% of this license or (at your option) any later version.
%% The latest version of this license is in
%%   http://www.latex-project.org/lppl.txt
%% and version 1.3 or later is part of all distributions of LaTeX
%% version 2003/12/01 or later.
%%
%% This work has the LPPL maintenance status "maintained".
%% 
%% The Current Maintainer of this work is Juergen Spitzmueller.
%%
%% Code repository and issue tracker: https://github.com/jspitz/jslectureplanner
%%
%%%%%%%%%%%%%%%%%%%%%%%%%%%%%%%%%%%%%%%%%%%%%%%%%%%%%%%%%%%%%%%%%%%%%%%%%%%%%%%%%%%%%%%%%%%%%%%%%%%

\documentclass[english]{article}
\usepackage[osf]{libertine}
\usepackage[scaled=0.76]{beramono}
\usepackage[T1]{fontenc}
\usepackage[latin9]{inputenc}

\usepackage{listings}
\lstset{basicstyle={\ttfamily}}
\usepackage{babel}
\usepackage{url}
\usepackage[
	    unicode=true,
	    bookmarks=true,bookmarksnumbered=false,bookmarksopen=false,
	    breaklinks=false,pdfborder={0 0 0},backref=false,colorlinks=false
	   ]{hyperref}
\hypersetup{%
	pdftitle={The biblatex-publist manual},
	pdfauthor={J�rgen Spitzm�ller},
	pdfkeywords={biblatex,publication list}
}

% Some semantic markup
\makeatletter
\def\jmacro{\@ifstar\@@jmacro\@jmacro}
\newcommand*\@marginmacro[1]{\marginpar{\small\texttt{#1}}}
\newcommand*\@jmacro[1]{\textbf{\texttt{#1}}}
\newcommand*\@@jmacro[1]{\@jmacro{#1}\@marginmacro{#1}}
\def\jcsmacro{\@ifstar\@@jcsmacro\@jcsmacro}
\newcommand*\@jcsmacro[1]{\@jmacro{\textbackslash{#1}}}
\newcommand*\@@jcsmacro[1]{\@@jmacro{\textbackslash{#1}}}
\newcommand*\joption[1]{\textbf{\texttt{#1}}}
\newcommand*\jfoption[1]{\texttt{#1}}
\newcommand*\jfmacro[1]{\texttt{#1}}
\newcommand*\jfcsmacro[1]{\jfmacro{\textbackslash{#1}}}
\newcommand*\bpl{\texttt{biblatex-publist}}
\newcommand*\bibltx{\texttt{biblatex}}
\makeatother

\reversemarginpar

% Conditional page breaks
\def\condbreak#1{%
	\vskip 0pt plus #1\pagebreak[3]\vskip 0pt plus -#1\relax}

\renewcommand{\lstlistingname}{Example}

%
%%
\begin{document}

\title{biblatex-publist}

\author{J�rgen Spitzm�ller%%
\thanks{Please report issues via \protect\url{https://github.com/jspitz/biblatex-publist}.}%
}

\date{Version 1.4dev, 2016/08/06}
\maketitle

\begin{abstract}
\noindent The \bpl\ package provides a \emph{biblatex bibliography style file}
({*}.bbx) for publication lists, i.\,e.\ a bibliography containing one's own
publications. The style file draws on \bibltx's \emph{authoryear} style per default
(which can be changed), but provides some extra features needed for publication lists,
such as the omission of the own name from author or editor data. The package requires at least version 3.4 of the \bibltx\ package\footnote{For \bibltx, see
\url{http://www.ctan.org/pkg/biblatex}.} and \texttt{biber} (the respective version as required by \bibltx).
\end{abstract}

\section{Aim of the package}

The \bpl\ package ships a \emph{biblatex bibliography style file} ({*}.bbx)
for a specific task: academic publication lists. Such lists, which are a central
part of the academic CV, contain all or selected publications of a specific
author, usually sorted by genre and year. Even though publication lists are actually
nothing else than (specific) bibliographies, they diverge from those in some
respects. Most notably, it is widespread practice to omit your own name in
your publication list and only list your co-authors, if there are any.
If you want to follow this practice, a normal bibliography style does not
produce the desired result.

Given the fact that maintaining a publication list is a routine task
in an academian's life, it is surprising how few specified solutions
exist to generate such lists (particularly from Bib\TeX\ data). For
traditional Bib\TeX, Nicolas Markey provided a specific Bib\TeX\
style file, \emph{publist.bst}%
\footnote{\url{http://www.lsv.ens-cachan.fr/~markey/BibTeX/publist/?lang=en};
see also \cite{ttb}.}, which helps a lot if you want to produce a publication
list with Bib\TeX. The \bpl\ package is the result of the aim of emulating
the features of \emph{publist.bst} with \bibltx's means. It partly draws on
Markey's conceptual ideas. Bug reports, comments and ideas are welcome.\filbreak


\section{Loading the package}

\subsection{Standard usage\label{sec:standard-usage}}

The standard way of using the package is to load the style file via
\begin{quote}
\begin{lstlisting}[language={[LaTeX]TeX},moretexcs={[1]{omitname}}]
\usepackage[bibstyle=publist]{biblatex}
\omitname[first name][von-part]{surname}
\end{lstlisting}
\end{quote}
The \jcsmacro*{omitname} macro (at least with the mandatory
\emph{surname} argument) needs to be given once. It informs the style
file which name it should suppress in the author\slash editor list
(usually yours). That is to say: For all of your publications where
you are the sole author or editor, the author\slash editor name
will be omitted completely, as in:
\begin{quote}
\textbf{2012.} Some recent trends in gardening. In: \emph{Gardening
Practice} 56, pp.~34--86.
\end{quote}
If there are co-authors\slash co-editors, your name will be filtered
out and the collaborators added in parentheses, as in:
\begin{quote}
\textbf{1987} (with John Doe and Mary Hall). Are there new trends
in gardening? In: \emph{Gardening Practice} 24, pp.~10--15.
\end{quote}


\subsection{Additional options}

Currently, the following additional options are provided (next to the
options provided by the \bibltx\ package itself%
\footnote{Please refer to the \bibltx\ manual \cite{bibltx} for those.}):
\begin{description}
\item [{\joption{omitname=<surname>}}]
\item [{\joption{omitfirstname=<first name>}}]
\item [{\joption{omitnameprefix=<von-part>}}] ~

This is an alternative to the \jcsmacro{omitname} macro
described in sec.~\ref{sec:standard-usage}. However, due to the
way bibliography options are implemented in \bibltx, this only works
if your name does not consist of non-ASCII characters. Hence, the
\jcsmacro{omitname} macro is the recommended way.

\item [{\joption{boldyear{[}=true|false{]}}}] default: \emph{true}.

By default, the year (or pubstate, if no year is given) is printed in bold face.
To prevent this, pass the
option \joption{boldyear=false} to \bibltx.

\item [{\joption{marginyear{[}=true|false{]}}}] default: \emph{false}.

With this option set to \joption{true}, the publication year (or pubstate) will
be printed in the margin once a new year starts. The option also has
the effect that all marginpars are printed ``reversed'', i.\,e.
on the left side in one-sided documents (via \jfcsmacro{reversemarginpar}).

\end{description}

\subsection{Auxiliary macros and lengths}\label{sec:auxmacros}

The appearance of the \emph{marginyear} is controlled by the
\jcsmacro*{plmarginyear} macro, which has the following default definition:
\begin{quote}
\begin{lstlisting}[language={[LaTeX]TeX},
                   moretexcs={[2]{providecommand,plmarginyear}}]
\providecommand*\plmarginyear[1]{%
  \raggedleft\small\textbf{#1}%
}
\end{lstlisting}\end{quote}
If you want to change the appearance, just redefine this macro via
\jfcsmacro{renewcommand{*}}.

The indendation of the bibliographic entries (lines > 1) can be adjusted by setting the length
\jmacro*{extralabel\-numberwidth} via \jfcsmacro{setlength} (default is \texttt{0pt}).
This might be needed for long bibliographies (> 99 entries) in order to adjust to the extra
space the item number needs.


\subsection{Using a different base style}\label{sec:basestyle}

By default, \bpl\ loads \bibltx's \emph{authoryear} style, and it has been written
to work with that style. However, it is possible to try a diffent base style, if
\emph{author\-year} does not fit your needs.

In order to do so, enter the following \emph{before} loading \bibltx:
\begin{quote}
	\begin{lstlisting}[language={[LaTeX]TeX},
	                   moretexcs={[2]{publistbasestyle}}]
	\newcommand*\publistbasestyle{<stylename>}
	\end{lstlisting}\end{quote}
where <stylename> is the name of the biblatex bibliography style (\emph{bbx}) you want to use, without the \emph{bbx} entension (e.\,g., \lstinline|\newcommand*\publistbasestyle{mla}|).

Note, though, that there is (and can be) no guarantee that \bpl\ will work with all styles, although it has been successfully tested with several. Be prepared to bump into \LaTeX\ errors and carefully check the output for correctness if you try a different base style.

\section{Localization}

Since the package draws on \bibltx, it supports localization. 
The following additional localization keys (\jfcsmacro{bibstrings})
are added by the package:
\begin{itemize}
\item \emph{with}: the preposition ``with'' that precedes the list of
co-authors.
\item \emph{parttranslationof}: the expression ``partial translation of''
for entries referring to partially translated work via \bibltx's
``related entries'' feature (see sec.~\ref{sec:partial-translations}).
\end{itemize}
Currently, these additional localization keys are available in the following
languages: English, French and German.%
\footnote{Please send suggestions for other languages to the package author.}

\section{Further Extensions}

The following extensions of standard \bibltx\ features are provided.


\subsection{Review bibliography type}\label{sec:review-bibliography-type}

Although a \emph{review} entry type is provided by \bibltx, this
type is treated as an alias for \emph{article}. The \bpl\ package
uses this entry type for a specific purpose: Foreign reviews of your
own work. It therefore defines a new bibliography environment \emph{reviews}
with a specific look (particularly as far as the author names are
concerned) and its own numbering; furthermore, it redefines the \emph{review}
bibliography driver. The purpose of this is that you can add other
people's reviews of your work to your publication list, while these
titles are clearly marked and do not interfere with the overall numbering
(see sec.~\ref{sec:example} for an example).


\subsection{Partial translations}\label{sec:partial-translations}

A new ``related entry'' type \emph{parttranslationof} is provided.
This is an addition to the \emph{translationof} related entry type
\bibltx\ itself provides. Please refer to the \bibltx\ manual \cite{bibltx}
on what ``related entries'' are and how to use them.


\section{An example}\label{sec:example}

Publication lists are usually categorized by genre (monographs, articles,
book chapters, etc.). For this task, the use of \jfmacro{refsections} is
suggested. Other possibilities were not tested extensively and might fail.

The suggested procedure is to maintain separate bib files for each
category, say \emph{mymonographs.bib}, \emph{myarticles.bib}, \emph{myproceedings.bib}.%
\footnote{But see sec.~\ref{sec:filtering} for an alternative.}
Then a typical file would look like example~\ref{example} (p.~\pageref{example}).

\begin{lstlisting}[caption={Typical document},
		   float,
		   frame=single,
		   label={example},
		   language={[LaTeX]TeX},
		   moretexcs={[4]{omitname,addbibresource,printbibliography,maketitle}}]
\documentclass{article}
\usepackage[T1]{fontenc}
\usepackage[latin9]{inputenc}

\usepackage{csquotes}% not required, but recommended
\usepackage[bibstyle=publist]{biblatex}
\omitname[John]{Doe}

\addbibresource{%
    mymonographs.bib,
    myarticles.bib,
    myproceedings.bib
}

\begin{document}

\title{John Doe's publications}
\date{\today}
\maketitle

\section{Monographs}
\begin{refsection}[mymonographs]
\nocite{*}
\printbibliography[heading=none]
\end{refsection}

\section{Proceedings}
\begin{refsection}[myproceedings]
\nocite{*}
\printbibliography[heading=none]
\end{refsection}

\section{Articles}
\begin{refsection}[myarticles]
\nocite{*}
\printbibliography[heading=none]
\end{refsection}

\end{document}
\end{lstlisting}


If you want to add other people's reviews of your work, add a section
such as the following:
\begin{quote}
\begin{lstlisting}[caption={Adding foreign reviews},
		  frame=single,
		  language={[LaTeX]TeX},
		  moretexcs={[3]{bibfont,subsubsection,printbibliography}}]
\subsubsection*{Reviews of my thesis}
\begin{refsection}[mythesis-reviews]
\renewcommand\bibfont{\small}
\nocite{*}
\printbibliography[heading=none,env=reviews]
\end{refsection}
\end{lstlisting}

\end{quote}
Note that the \jfcsmacro{printbibliography} option
\joption{env=reviews}  is crucial if you want to use the specifics
\bpl\ defines for reviews (see sec.~\ref{sec:review-bibliography-type}).


\section{Filtering\label{sec:filtering}}

If you have a bibliographic database consisting not only of your own
publications, you can extract yours with the bibliography filter \joption{mine},
which has to be passed to \jfcsmacro{printbibliography}, as in:
\begin{quote}
\begin{lstlisting}[caption={Using a bibliography filter},
		  frame=single,
		  language={[LaTeX]TeX},
		  moretexcs={[1]{printbibliography}}]
\begin{refsection}[mybibliography]
\nocite{*}
\printbibliography[heading=none,filter=mine]
\end{refsection}
\end{lstlisting}

\end{quote}
Of course, you can also use other filter possibilities provided by
\bibltx, such as filtering by type or by keyword. So if you want
to extract all of your articles from a larger database with entries
of diverse type and authors, specify:
\begin{quote}
\begin{lstlisting}[language={[LaTeX]TeX},
		   moretexcs={[1]{printbibliography}}]
\printbibliography[heading=none,filter=mine,type=article]
\end{lstlisting}
\end{quote}


\section{Sorting\label{sec:sorting}}

The sorting of the items is done via \bibltx's sorting mechanism
(please refer to the \bibltx\ manual for details). By default,
\bpl\ uses the \jfmacro{ydnt} scheme, which sorts hierarchically
by year (descending), name and title (both ascending). You can switch to another
scheme via \bibltx's \jfmacro{sorting} option either globally (if you pass
\joption{sorting=<scheme>} to the \bibltx\ options) or locally (if you pass
\joption{sorting=<scheme>} to the \jfcsmacro{printbibliography} options).

For convenience, \bpl\ provides 3 additional sorting schemes, which might
be particularly useful for sorting talks:
\begin{itemize}
 \item \jmacro{ddnt}: Sort by full \textbf{d}ate (\textbf{d}escending),
                      \textbf{n}ame and \textbf{t}itle (both ascending).
 \item \jmacro{ydmdnt}: Sort by \textbf{y}ear (\textbf{d}escending),
                        \textbf{m}onth, \textbf{d}ay, \textbf{n}ame and \textbf{t}itle
                        (all ascending).
 \item \jmacro{dnt}: Sort by full \textbf{d}ate, \textbf{n}ame and \textbf{t}itle
                        (all ascending).
\end{itemize}
That is, to sort your talks in descending order by full date in your CV, use:
\begin{quote}
\begin{lstlisting}[language={[LaTeX]TeX},
		   moretexcs={[1]{printbibliography}}]
\printbibliography[heading=none,sorting=ddnt]
\end{lstlisting}
\end{quote}

\clearpage

\section{Revision Log}

\begin{description}
\item [{V.~1.4 (forthcoming):}]~
	
	\begin{itemize}
		\item Assure the margin text always starts uppercased (relevant for pubstates).
		\item Minor corrections to the manual.
	\end{itemize}
\item [{V.~1.3 (2016-08-06):}]~

\begin{itemize}
	\item It is now possible to change the base style that is used by \bpl. See sec.~\ref{sec:basestyle}.
	\item Proper sorting of pubstates.
	\item Add possibility to increase the indentation of items (by means of the length  \jmacro{extralabelnumberwidth}). See sec.~\ref{sec:auxmacros}.
	\item Use \jfoption{pagetracker=true} instead of \jfoption{pagetracker=spread}
	      by default (avoids warning, no change in functionality).
\end{itemize}

\item [{V.~1.2 (2016-05-12):}]~

	\begin{itemize}
		\item Accomodate to the backwards-incompatible changes of \bibltx~3.4\\
		(\jfoption{prefixnumber} $\Rightarrow$ \jfoption{labelprefix},
		\jfcsmacro{ifempty} $\Rightarrow$ \jfcsmacro{ifdefvoid}).
		This version of \bibltx\ is now required.
	\end{itemize}

\item [{V.~1.1 (2016-03-09):}]~

\begin{itemize}
\item Adapt to the \jfcsmacro{Declare*Name} changes of \bibltx~3.3.
      Since \bibltx~3.3 introduced backwards-incompatible changes that
      affect \bpl, this version of \bibltx\ is now required.
\end{itemize}

\item [{V.~1.0~(2015-01-04):}]~

\begin{itemize}
\item Add portmanteau *.cbx file to allow loading \bpl\ also via 
      the \jfoption{style} option (next to \jfoption{bibstyle}).
\end{itemize}

\item [{V.~0.9~(2014-03-13):}]~

\begin{itemize}
\item Fix problem with multi-token names.
\item Support name prefix in \jcsmacro{omitname}.
\item Support pubstate.
\end{itemize}

\item [{V.~0.8~(2013-08-16):}]~

\begin{itemize}
\item Add custom sorting schemes \jmacro{ddnt}, \jmacro{ydmdnt} and \jmacro{dnt}
      (see sec.~\ref{sec:sorting}).
\item Revise the documentation.
\end{itemize}\condbreak{2\baselineskip}

\item [{V.~0.7~(2013-07-25):}]~

\begin{itemize}
\item Support full dates.
\end{itemize}
\item [{V.~0.6~(2013-07-21):}]~

\begin{itemize}
\item Fix numbering with recent \bibltx\ versions.
\end{itemize}
\item [{V.~0.5~(2013-05-03):}]~

\begin{itemize}
\item Fix numbering if \jfcsmacro{printbibliography} is used
multiple times within the same or without any \jfmacro{refsection}.
\end{itemize}
\item [{V.~0.4~(2012-10-30):}]~

\begin{itemize}
\item More robust name parsing (especially for names with non-ASCII characters
encoded with \LaTeX{} macros). The code was kindly suggested by Enrico
Gregorio.%
\footnote{Cf. \url{http://tex.stackexchange.com/questions/79555/biblatex-bibliographyoption-with-braces}.%
}
\item Add \jcsmacro{omitname} command (see sec.~\ref{sec:standard-usage}).
\item Support \joption{firstinits} option.
\end{itemize}
\item [{V.~0.3~(2012-10-23):}]~

\begin{itemize}
\item Bug fix: Add missing ``and'' if omitted name was last minus one.
\item Bug fix: Fix output with ``et al.'' if omitted name is first and
\emph{liststop} is 1.
\item Set \joption{maxnames} default to 4.
\item Add filter possibility (see sec.~\ref{sec:filtering}).
\item Add French localization.
\item Some corrections to the manual.
\end{itemize}
\item [{V.~0.2~(2012-10-21):}] Initial release to CTAN.
\end{description}

\section{Credits}

Thanks go to Enrico Gregorio (egreg on \emph{tex.stackexchange.com})
for helping me with correct name parsing (actually, the code the package
uses is completely his), Marko Budi�i\'{c}, Clea F. Rees and Yannick Kalff
for testing and bug reports, Nicolas Markey for \emph{publist.bst} and of
course Philipp Lehman (not only) for \bibltx.

\begin{thebibliography}{1}
\bibitem{bibltx}Lehman, Philipp (with Audrey Boruvka, Philip Kime
and Joseph Wright): \emph{The biblatex Package. Programmable Bibliographies
and Citations}. March 3, 2016.
\url{http://www.ctan.org/pkg/biblatex}.

\bibitem{ttb}Markey, Nicolas: \emph{Tame the BeaST. The B to X of
BibTEX}. October 11, 2009.
\url{http://www.ctan.org/pkg/tamethebeast}.
\end{thebibliography}

\end{document}
